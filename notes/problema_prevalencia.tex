\documentclass[a4paper, notitlepage, 10pt]{article}
\usepackage{geometry}
\fontfamily{times}
\geometry{verbose,tmargin=30mm,bmargin=25mm,lmargin=25mm,rmargin=25mm}
% \pagestyle{empty}
% end configs
\usepackage{setspace,relsize}               
\usepackage{moreverb}                        
\usepackage{url}
\usepackage{hyperref}
\hypersetup{colorlinks=true,citecolor=blue}
\usepackage{amsmath}
\usepackage{mathtools} 
\usepackage{amsthm}
\usepackage{amssymb}
\usepackage{indentfirst}
\usepackage{todonotes}
\usepackage[authoryear,round]{natbib}
\bibliographystyle{apalike}
\usepackage[pdftex]{lscape}
\usepackage[utf8]{inputenc}
\usepackage[portuguese]{babel}

% Title Page
\title{\vspace{-9ex}\centering \bf Estimando a prevalência de uma doença a partir de um teste diagnóstico}
\author{
Claudio Struchiner \& Luiz Max de Carvalho \\
Escola de Matemática Aplicada (EMAp), Get\'ulio Vargas Foundation.
% Program for Scientific Computing (PROCC), Oswaldo Cruz Foundation.\\
% Institute of Evolutionary Biology, University of Edinburgh.\\
}
\DeclareMathOperator*{\argmin}{arg\,min}
\DeclareMathOperator*{\argmax}{arg\,max}
\newtheorem{theorem}{Theorem}[]
\newtheorem{proposition}{Proposition}[]
\newtheorem{remark}{Remark}[]
\setcounter{theorem}{0} % assign desired value to theorem counter
\begin{document}
\maketitle

% \begin{abstract}
% 
% Key-words: ;; ; ; . 
% \end{abstract}

% \section*{Background}

\section*{Introdução}

Suponha que desejamos estimar a proporção $\theta$ de indivíduos infectados com um determinado patógeno em uma população.
Suponha ainda que dispomos de um teste laboratorial, que produz o resultados $r = \{-, +\}$ indicando se o indivíduo ($y_i$) é livre ($0$) ou infectado ($1$).
Se o teste fosse perfeito, poderíamos escrever a probabilidade de observar $y = \sum_{i =1}^n y_i$ testes positivos em $n$ testes realizados como\footnote{Porquê?}
\begin{equation}
\label{eq:prev_simple}
 \operatorname{Pr}\left( y \mid \theta, n \right)= \binom{n}{y} \theta^y (1-\theta)^{n-y}. 
\end{equation}


Infelizmente, o teste não é perfeito, acertando o diagnóstico com probabilidades fixas da seguinte forma
\begin{align}
%  \operatorname{Pr}\left(\right)
 \operatorname{Pr}\left(\ r = + \mid y_i = 0 \right) &:= 1-u,\\
 \operatorname{Pr}\left( r = - \mid y_i = 1 \right) &:= 1-v,
\end{align}
de modo que agora escrevemos\footnote{Exercício bônus: mostre porquê.}
\begin{equation}
  \operatorname{Pr}\left(r = + \mid \theta, u, v \right) := \theta ( 1- v) + (1-\theta)u,
\end{equation}
de modo que podemos reescrever a probabilidade em~(\ref{eq:prev_simple}):
\begin{equation}
 \operatorname{Pr}\left(y \mid \theta, n, u, v\right) = \binom{n}{y} \left[ u + \theta ( 1- (u  + v)) \right]^{y} \left[ 1 -u - \theta (1 -(u + v))\right]^{n-y}.
\end{equation}


\section*{Problema(s)}
\begin{itemize}
 \item[a)] Escolha e justifique uma distribuição~\textit{a priori} para $\theta$ -- lembre-se que neste exercício $u$ e $v$ são fixos;
 \item[b)] Derive $\operatorname{Pr}(\theta \mid y, n, u, v)$;
 \item[c)] Suponha que $y = 4$ e $n = 5000$.
 Qual a média~\textit{a posteriori} de $\theta$?
 Produza intervalos de credibilidade de $80$, $90$ e $95$\% para $\theta$. 
 \item[d)] \textbf{Bônus}. Que melhorias você faria neste modelo? Que outras fontes de incerteza estão sendo ignoradas?
\end{itemize}

\section*{Dicas}
\begin{itemize}
 \item Lembrem-se de justificar~\textbf{todas} as suas respostas, tanto matematica quanto estatisticamente. 
 Isto inclui a escolha de métodos numéricos, se estes forem necessários;
 \item O tópico abordado aqui é bastante conhecido e existe farta literatura.
 As palavras-chave são: diagnostic tests, sensitivity, specificity, Bayesian estimation.
 Por enquanto, não vamos recomendar os artigos pertinentes para que o problema não perca a graça uma vez que vocês vejam a solução. 
 Quando formos discutir as abordagens para o problema, revelaremos alguns dos artigos que podem ser consultados;
 \item Vocês podem consultar os capítulos 7.2 e 7.3 de~\cite{DeGroot2012} para o item a) e também o capítulo 3 de~\cite{McElreath2020} para os outros itens.
 \item Procurem se divertir com os problemas nas notas de rodapé; eles valem o esforço e envolvem apenas relembrar conceitos de probabilidade.
\end{itemize}


\bibliography{../biblio/statmodelling}

% \begin{figure}[!ht]
% \centering
% \includegraphics[width=\textwidth, height = 15cm]{figures/}
% \caption{\textbf{}.
% }
% \label{fig:}
% \end{figure}
%%
% \begin{figure}
% \hfill
% \subfigure[Title A]{\includegraphics[width=5cm]{img1}}
% \hfill
% \subfigure[Title B]{\includegraphics[width=5cm]{img2}}
% \hfill
% \caption{\textbf{}.
% }
% \label{fig:}
% \end{figure}
\end{document}          
